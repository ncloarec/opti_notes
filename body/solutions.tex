\begin{theorem}

Soit $f$ une fonction réelle $\alpha$-fortement convexe et $L$-lipschitzienne sur
$E$. Soit $\xopt:= \argmin_{x\in E} f(x)$. On a le résultat suivant:
\begin{equation}
f(\xhat)-f(\xopt) \leq \frac{2 L^2}{\alpha T}
\end{equation}
\end{theorem}

\begin{proof}
Pour $t=1$ à $T$, introduisons le pas:
\begin{equation}
  \mu_t := \frac{2}{\alpha t}
\end{equation}
En utilisant l'identité classique $2\ps{x}{y} = \norme{x+y}^2-\norme{x}^2-\norme{y}^2$, on a pour $t=0$ à $T-1$:
\begin{align*}
f(\xhat)-f(\xopt) &\leq f(x_t)-\prt{f(x_t) + \transpose{u_t}\prt{\xopt-x_t}
+ \frac{\alpha}{2} \norme{\xopt-x_t}^2}\\
&= -\transpose{u_t}\prt{\xopt-x_t}
- \frac{\alpha}{2} \norme{\xopt-x_t}^2\\
&= -\frac{\step}{\step} \cdot \transpose{u_t}\prt{\xopt-x_t}
- \frac{\alpha}{2} \norme{\xopt-x_t}^2\\
&= \frac{1}{2 \step }\prt{\norme{\step \cdot u_t}^2+\norme{\xopt-x_t}^2-\norme{\step \cdot u_t+\xopt-x_t}^2
}- \frac{\alpha}{2} \norme{\xopt-x_t}^2\\
&= \frac{1}{2 \step }\prt{\norme{\step \cdot u_t}^2+\norme{\xopt-x_t}^2-\norme{ \xopt-y_t}^2
}- \frac{\alpha}{2} \norme{\xopt-x_t}^2\\
&\leq  \frac{1}{2 \step }\prt{\step^2\cdot L^2+\norme{\xopt-x_t}^2-\norme{\xopt-x_{t+1}}^2
}- \frac{\alpha}{2} \norme{\xopt-x_t}^2\\
&= \frac{\step L^2}{2} + \underbrace{\prt{\frac{1}{2 \step } - \frac{\alpha}{2}}}_{\frac{\alpha (t-1)}{4}} \norme{\xopt-x_t}^2
- \underbrace{\frac{1}{2 \step }}_{\frac{\alpha (t+1)}{4}} \norme{\xopt-x_{t+1}}^2
\end{align*}

En utilisant la définition de $\xhat$, la convexité de $f$ et l'inégalité d'au-dessous pour tout $t$, on a successivement:
\begin{align*}
  f(\xhat)-f(\xopt) &= f\prt{\sum_{t=0}^T \frac{2t}{T(T+1)} x_t}-f(\xopt)\\
  &\leq  \sum_{t=0}^T \frac{2t}{T(T+1)} \prt{f(x_t)-f(\xopt)}\\
  &\leq  \sum_{t=0}^T \frac{2t}{T(T+1)} \prt{\frac{\step L^2}{2}+\frac{\alpha (t-1)}{4} \cdot \norme{\xopt-x_t}^2-\frac{\alpha (t+1)}{4} \cdot  \norme{\xopt-x_{t+1}}^2}\\
  &\leq  \underbrace{\sum_{t=0}^T \frac{t}{T(T+1)} \step L^2}_{:=A}+\underbrace{\sum_{t=0}^T \frac{\alpha t}{2T(T+1)}  \prt{(t-1) \cdot \norme{\xopt-x_t}^2-(t+1) \cdot  \norme{\xopt-x_{t+1}}^2}}_{:=B}\\
\end{align*}

En posant $\delta_t := t(t-1) \cdot \norme{\xopt-x_t}^2$, on peut réécrire la somme
 de gauche:
\begin{equation*}
B= \frac{\alpha}{2T(T+1)} \sum_{t=0}^T \prt{\delta_t-\delta_{t+1}}
=\frac{\alpha}{2T(T+1)} \prt{\delta_1-\delta_{T+1}}
\leq 0
\end{equation*}

Montrons maintenant que $A \leq \frac{2 L^2}{\alpha T}$. En utilisant la définition
 du pas $\mu_t$, on a:
 \begin{equation*}
   A= \frac{L^2}{T(T+1)} \sum_{t=0}^T \frac{2t}{\alpha (t+1)}
   = \frac{2L^2}{\alpha T} \cdot  \frac{1}{(T+1)}\sum_{t=0}^T \underbrace{\frac{t}{t+1}}_{\leq 1} \leq \frac{2L^2}{\alpha T}
 \end{equation*}
d'où le résultat.
\end{proof}

Un exemple très important est celui des fonctions $\alpha$-fortement convexe et
 $\beta$ régulière. Remarquons que nécessairement $\alpha \leq \beta$.

 \section{Cas de fonctions $\alpha$-fortement convexe et
  $\beta$ régulière}

  Soit $f: \R^d \mapsto \R$ une fonction $\alpha$-fortement convexe et
   $\beta$ régulière. On a donc pour tout $x, x_0$ appartenant à $\R^d$:
   \begin{description}
     \item[$\beta$-régularité] $\displaystyle f(x) \leq f(x_0)+ \transpose{\grad f(x_0)}\prt{x-x_0}+ \frac{\beta}{2}\norme{x-x_0}^2$
     \item[$\alpha$-fortement convexe] $\displaystyle f(x_0)+ \transpose{\grad f(x_0)}\prt{x-x_0}+ \frac{\alpha}{2}\norme{x-x_0}^2 \leq  f(x) $
   \end{description}

   Plusieurs cas :
   \begin{itemize}
     \item Si $\alpha=\beta$, cela signifie que $f$ est quadratique;
     \item si $\alpha \neq \beta$, alors nécessairement $\alpha \leq \beta$.
     \item Si $\alpha \approx \beta$, alors on peut confondre la courbe de $f$ et
     la courbe de $\beta$ et on peut alors trouver le minimum très facilement.
   \end{itemize}

   \begin{theorem}
En reprenant les hypothèses sur $f$ introduites au-dessus, on a le résultat suivant
\begin{equation}
\norme{\xhat - \xopt}^2 \leq \e^{-\frac{T}{K} \cdot \norme{x_0-\xopt}}
\end{equation}
où $K := \frac{\beta}{\alpha}$.
   \end{theorem}

\begin{proof}
Montrons tout d'abord l'égalité suivante:
\begin{equation}
  f(\xhat)- f(\xopt) = \int_{0}^{1} \transpose{\grad f\prt{\xopt+t(\xhat-\xopt)}}
   \cdot \prt{\xhat- \xopt} \dt
\end{equation}

En introduisant la fonction auxiliaire $\phi(t):= f\prt{\xopt+t(\xhat-\xopt)}$ définie sur $\intff{0}{1}$, on a :
\begin{equation*}\label{eq:int_form}
  \phi(1)-\phi(0)= \int_0^1 \phi^{\prime}(t) \dt = \int_{0}^{1} \transpose{\grad f\prt{\xopt+t(\xhat-\xopt)}}
   \cdot \prt{\xhat- \xopt} \dt
\end{equation*}
ce qui montre \eqref{eq:int_form} .

En utilisant la forme intégrale pour $f(\xhat)- f(\xopt)$, on a donc succcessivement:
\begin{align*}
f(\xhat)- f(\xopt) &= \int_{0}^{1} \transpose{\grad f\prt{\xopt+t(\xhat-\xopt)}}
 \cdot \prt{\xhat- \xopt} \dt\\
 &\leq \int_{0}^{1} \norme{\grad f\prt{\xopt+t(\xhat-\xopt)}}
  \cdot \norme{\xhat- \xopt} \dt
\end{align*}
\end{proof}
